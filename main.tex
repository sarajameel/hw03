\documentclass[a4paper,12pt]{article}
\usepackage{color}
\usepackage{fancyvrb}
\usepackage{geometry}

\setlength\parindent{0pt}

\title{CS 224 Object Oriented Programming and Design Methodologies}
\author{Assignment 03}
\date{Out: 24 September, Due: 5 Oct, 1830h}

\begin{document}
\maketitle

\section{Task}
For this assignment you will be creating a package delivery system. The system contains trucks with each truck having a certain load, i.e. a certain number of packages, that the truck has to deliver and then return. Every truck has certain resources at its disposal. If a truck has insufficient resources, it will not be able to successfully deliver its load and will therefore not begin its journey.

The trucks have a capacity of 50 liters of diesel and every truck that begins its delivery must fill up before leaving. One liter of diesel costs \$ 2.73. Each truck can carry 12 to 20 boxes of arbitrary length, width, and height which range from 5 to 30 inches. 

Your task is to read truck information from a given file, assign a random load to each truck, output loading and unloading information of each truck to screen, dispatch the trucks that can deliver, and then write the final state of each truck to a new file.

\subsection{Input and Output}

Truck information will be given in a file similar to the attached, {\tt Drivers.txt}. It contains the information of each truck on 5 lines. For example, the first 5 lines in the file are:
\begin{Verbatim}[frame=single]
Elton John
34
218
9
7
\end{Verbatim}
and they correspond to a truck driven by {\tt Elton John} which currently holds {\tt 34} liters of diesel in its tank. The driver has total funds of \$ 218 and the truck covers {\tt 9} km per liter when empty and {\tt 7} km per liter when loaded.

You will assign a random load to each truck and display the number of boxes and dimensions of each box to screen along with the name of the driver. When the truck unloads, you will again display this information.

Lastly, you will write the final state of each truck to a new file, {\tt Trip.txt}. For trucks that did not deliver, the final state will be the same as the initial state that as read from file.

\subsection{Hints}
\begin{itemize}
\item Think in terms of the things you have to model. These will be the classes in your program. Candidate classes in this situation are {\tt Truck} and {\tt Box}.
\item Each {\tt Truck} object will be assigned an arbitrary number ($\in [12,20]$) of {\tt Box} objects. This can be modeled with a dynamic array of {\tt Box} objects contained in each {\tt Truck}.
\item The {\tt Truck} class needs a {\tt load} method which assigns it a random load and prints the load information as described above.
\item The {\tt Truck} class correspondingly needs an {\tt unload} method which deallocates the contained {\tt Box} objects and prints the load information as described above.
\item As you do not know the number of trucks whose information is contained in the input file, you will need a dynamic array of {\tt Truck}'s.
\end{itemize}

\section{Coding Tips}

Some important points:

\begin{itemize}
\item The provided sample code is for your benefit. If you are going to use it, understand how it works. You are not required to use it.
\item The sample code uses {\tt struct}s. You need to change it appropriately to use {\tt class}es. A general rule of thumb is that all attributes in a class are {\tt private} by default.
\item Where necessary, declare your own functions inside classes. Decide carefully for each function if it should be {\tt private} or {\tt public}.
\item You do not need to follow the sample code exactly. You may make changes where you see fit provided that it makes sense.
\item You need to define separate {\tt *.h} and {\tt *.cpp} files for all the classes.
\item Take care to appropriately deallocate any and all dynamically allocated memory.
\end{itemize}

\begin{center}
  -- The End --
\end{center}

\end{document}